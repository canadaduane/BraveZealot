\chapter{Experiences and Time Spent}\label{chap:exp}
\section{Duane's Summary}
The majority of my time this week was spent optimizing our $A^*$ algorithm.  Our original implementation was entirely in Ruby, but it turned out to be too slow for discretization below 40 meters or so.  The new code is a Ruby extension written in C, and performs much better (1/2 second compared with 45 seconds on a $1000\times 1000$ grid).
\par
I also refactored our agent code to use state transitions for this lab.  Rather than using several different Agent classes for each task (e.g. a `dummy' agent vs a `smart' agent), we unified our code into a single agent that could make decisions regarding states.  Once this refactoring was in place, Jon Brandenburg specialized our agents for sniper and decoy capabilities.
\par
I spent about 18 hours on this project, divided as follows:
\begin{itemize}
    \item 8 hours implementing a fast $A^*$ algorithm in C
    \item 3 hours refactoring our agents to use state information
    \item 3 hours building a new Phttp://narray.rubyforge.org/DF output of our map with tank and flag positions
    \item 2 hours debugging and passing off with another team in the CS Sports lab.
    \item 2 hours writing this report and formatting it in \LaTeX
\end{itemize}

\section{Michael's Summary}
This week I started off trying to debug why our agents from the last lab kept getting stuck after capturing the enemy flag. I spent part of Sunday and Monday night tracing the problem through our code
and finally found it late Monday night.  The problem turned out to be related to caching the search path
and not re-calculating it when we had purposely cleared the cached search path.
\par
Later in the week I got to work on fixing up our PD controller to be more accurate for following paths.  This task - like the task above - was not directly related to the passof for this lab, but was a necessary step to making sure our team would be ready for the final lab.
\par
Finally towards the latter part of the week I spent my time on the aiming/shooting code.  I investigated 6 different methods of calculating the intersection of vectors.  I tried several different methods of treating the vectors as parametric equations as well as solving the problem as a system of linear equations and even some trigonometry.  In the end the a simple iterative approach proved to be the fastest while maintaining the same level of accuracy as the others.
\par
At the end of the week I had spent a total of 28 hours in the following areas:
\begin{itemize}
    \item 4 hours fixing the bug from last lab.
    \item 4 hours working on the PD controller to get better path following.
    \item 6 hours researching vector intersection methods
    \item 8 hours implementing several aim/shoot ideas
    \item 4 hours meeting with the team to passoff the lab
    \item 2 hours compiling report materials
\end{itemize}

\section{Jon's Summary}
I spent roughly five hours writing and testing all of the clay pigeons.