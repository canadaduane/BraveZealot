\chapter{Experiences and Time Spent}\label{chap:exp}
\section{Duane's Summary}
My assignment for this project was primarily to implement the Kalman filter and to show its output in PDF form for the report.  It was actually quite fun to turn the Kalman math into code using Ruby's NArray gem.  Because Ruby supports operator overloading, the matrix math was nearly as easy to code up as any algebraic equation.  We had to tweak the timing variable a bit so that our $t+1$ matrices properly represented the ``next iteration'' of the cycle.
\par
The ``screenshots'' in this report were produced using PDF output.  Part of my contribution was to generate the non-Kalman and Kalman paths for this output.
\par
I spent about 11 hours on this project, divided as follows:
\begin{itemize}
    \item 5 hours implementing the Kalman Filter in Ruby
    \item 3 hours merging code with Michael and passing off with another team in the lab
    \item 2 hours writing new PDF output with kalman paths
    \item 1 hour taking screenshots of the kalman paths
\end{itemize}

\section{Michael's Summary}
This week I started off trying to debug why our agents from the last lab kept getting stuck after capturing the enemy flag. I spent part of Sunday and Monday night tracing the problem through our code
and finally found it late Monday night.  The problem turned out to be related to caching the search path
and not re-calculating it when we had purposely cleared the cached search path.
\par
Later in the week I got to work on fixing up our PD controller to be more accurate for following paths.  This task - like the task above - was not directly related to the passof for this lab, but was a necessary step to making sure our team would be ready for the final lab.
\par
Finally towards the latter part of the week I spent my time on the aiming/shooting code.  I investigated 6 different methods of calculating the intersection of vectors.  I tried several different methods of treating the vectors as parametric equations as well as solving the problem as a system of linear equations and even some trigonometry.  In the end a simple iterative approach proved to be the fastest while maintaining the same level of accuracy as the others.
\par
At the end of the week I had spent a total of 28 hours in the following areas:
\begin{itemize}
    \item 4 hours fixing the bug from last lab.
    \item 4 hours working on the PD controller to get better path following.
    \item 6 hours researching vector intersection methods
    \item 8 hours implementing several aim/shoot ideas
    \item 4 hours meeting with the team to passoff the lab
    \item 2 hours compiling report materials
\end{itemize}

\section{Jon's Summary}
I spent roughly five hours writing and testing all of the clay pigeons.