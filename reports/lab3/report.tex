\documentclass[letterpaper,12pt]{report}
%
%--------------------   start of the 'preamble'
%
\usepackage{graphicx,amssymb,amstext,amsmath,url}
%
%%    homebrew commands -- to save typing
\newcommand\etc{\textsl{etc}}
\newcommand\eg{\textsl{eg.}\ }
\newcommand\etal{\textsl{et al.}}
\newcommand\Quote[1]{\lq\textsl{#1}\rq}
\newcommand\fr[2]{{\textstyle\frac{#1}{#2}}}
\newcommand\miktex{\textsl{MikTeX}}
\newcommand\comp{\textsl{The Companion}}
\newcommand\nss{\textsl{Not so Short}}
%
%---------------------   end of the 'preamble'
%

\begin{document}
%-----------------------------------------------------------
\title{Multi-Agent Lab \\
{\large \textbf{CS 470 Lab 3}}}
\author{Duane Johnson, Michael Ries and Jon Brandenburg}
\maketitle
%-----------------------------------------------------------
\begin{abstract}
This lab is all about coordination and higher-level algorithms that can be used to control agents.
Specifically this lab asks the students to implement a decoy tactic in which the enemy tanks are kept
busy shooting at a decoy while a second tank aims and shoots the enemy tanks.  In addition to the
added complexity of the decoy-sniper coordination the students are also required to use search paths to
find their way through a maze and get to a location.
\end{abstract}
%-----------------------------------------------------------
\tableofcontents
%-----------------------------------------------------------
\chapter{Lab Context}\label{chap:context}
\section{Summary}
The authors each developed and ran the software below on laptops.  The laptops were unix-based systems; however, these results should be easily duplicated on a Windows machine.

\section{Hardware}
\begin{itemize}
    \item Macbook Pro (32 bit) / 2.4 GHz Intel Core 2 Duo / 2 GB RAM
    \item Lenovo T400 (32 bit) / 2.2 GHz Intel Core 2 Duo / 2 GB RAM
    \item HP Pavilion (64 bit) / 2.5 GHz Intel Core 2 Duo / 4 GB RAM
\end{itemize}

\section{Software}
\begin{itemize}
    \item Mac OS X 10.5
    \item Ubuntu 9.04
    \item bzFlag 2.0.7 with patched bzrobots client\footnote{Supplied by CS470 in \textsl{Install Instructions}}
    \item Ruby Interpreter, version 1.8.7 (or 1.9.1)
    \item EventMachine Library\footnote{\url{http://rubyeventmachine.com/}} for Ruby
    \item PDF-Writer Library\footnote{\url{http://ruby-pdf.rubyforge.org/pdf-writer/}} for Ruby
    \item Rake Library\footnote{\url{http://rake.rubyforge.org/}} for Ruby Unit Tests
    \item NArray and NMatrix \footnote{\url{http://narray.rubyforge.org/}} for Ruby
\end{itemize}

\chapter{Experiences and Time Spent}\label{chap:exp}
\section{Duane's Summary}
Michael took the lead on this project by implementing the two most challenging search algorithms of those assigned---$A^*$ and \texttt{greedy best first}.  My portion was to implement \texttt{depth first}, \texttt{breadth first}, and \texttt{iterative deepening} search.  I also spent some time refactoring from our last week's work since many of the shortcuts we took before the deadline compromised the quality of the code.
\par
The \texttt{depth first} and \texttt{breadth first} algorithms were fairly straight-forward to implement---the most interesting part was the insight (from Russell and Norvig) that these two algorithms can both be implemented with the same code, but different queueing mechanisms.  With that insight, we chose to create a single search algorithm that would take a \texttt{queue} or \texttt{stack} object as a parameter.  \texttt{Breadth first} search corresponded with the \texttt{queue} object and \texttt{depth first} with the \texttt{stack}.
\par
Implementing the \texttt{iterative deepening} search was less straight-forward.  While on the surface it appears that a depth-limited search inside an iteratively deepening outer loop is sufficient to implement \texttt{iterative deepening}, it did not appear to be the case.  Instead, we needed each node to keep track of its predecessors and to consider those predecessors as the ``closed set'' during each inner iteration.  This is not the same as depth-limited search inside an outer loop because a depth-limited search will retain its ``closed set'' when in fact it should be forgetting certain paths that it has already tried (but again, it should not be revisiting nodes in its line of predecessors).
\par
I spent about 16 hours on this project, divided as follows:
\begin{itemize}
    \item 6 hours cleaning up and refactoring code from last week
    \item 3 hours implementing the search code mentioned above
    \item 2 hours debugging and passing off with another team in the CS Sports lab.
    \item 5 hours writing this report and formatting it in \LaTeX
\end{itemize}

\section{Michael's Summary}
This lab was intensely interesting.  On Tuesday night way before we started feeling the pressure I was up until 3:30 AM trying to get $A^*$ finalized just because I wanted to see the gnuplot file.  We ended up getting the gnuplot files working in general on Tuesday night and finished getting $A^*$ optimal on Wednesday.  The rest of the week we finished the other algorithms and got them cleaned up and tried to make them faster.  I focused on the discretization of the world and implementing the search algorithms that use heuristics ($A^*$ and \texttt{greedy best first}).
\par
While I was working on the discretization of the map I had the chance to try several different methods of discovering if the center of a chunk was inside of an obstacle.  This is an easy computation for squares, but somewhat more difficult for diamonds.  Also I wasn't able to find any guarantees in the documentation of our labs that we never have obstacles with less than 4 or more than 4 sides so the algorithm needed to work for any polygon.  I ended up finding a cool technique in which you draw vectors and take the cross-product of the vector representing a side of the obstacle and the vector between the corner of the obstacle and the point of interest.  This cross-product will have the same sign for all sides of a convex polygon if the point is inside the object.  If the point is outside the obstacle then the sign will change and you can quit checking.  This ended up being really fast because there was no trigonometric functions which are costly for processors to execute.
\par
The night I stayed up late working on $A^*$ I made a costly error of assuming that the problem was in the execution of the algorithm.  This meant that I spent hours of time looking at the algorithm trying to figure out what was wrong before I realized that the problem was my bookkeeping.  The fringe list was stored with references to each chunk on the map so when I re-encountered that same node its path-cost and heuristic were updated and changed which affected when it would be popped off.  Once I found the problem the algorithm magically started to work!
\par
In the end I spent a total of about 20 hours split across the following tasks:
\begin{itemize}
    \item 4 hours of re-working code from last lab to try to make our ``smart'' agents notice when the position of the flag has changed.
    \item 2 hours playing with algorithms to find if a point is inside an obstacle
    \item 4 hours of discretizing the map
    \item 3 hours implementing $A^*$
    \item 1 hour implementing \texttt{greedy best first}
    \item 2 hours finalizing and integrating code with Duane
    \item 4 hours passing off and preparing report material
\end{itemize}
\chapter{Details}\label{chap:details}
\section{Search and Analysis}\label{sec:search}
In the following graphs, the light blue (cyan) traces represent ``nodes that were considered'' along the way to a solution, and the red trace represents the ``final solution''.  The number of nodes popped from the stack or queue during each algorithm is noted, along with cost or distance of the path.

\subsection{Depth First Search}
The \texttt{depth first} search was the wildest of the bunch (see Figure ~\ref{fig:df}), and clearly not optimal: its ``solution'' goes all over the map and finally settles on a path that winds back and forth across half of the map.  Nodes popped: 2,411.  Distance of the path: 12,154 meters.

\begin{figure}\label{fig:df}
\begin{center}\centering
\includegraphics[width=\textwidth]{df1.png}
\caption{Depth first grows}
\end{center}
\end{figure}

\begin{figure}
\begin{center}\centering
\includegraphics[width=\textwidth]{df2.png}
\caption{Depth first finds a solution}
\end{center}
\end{figure}

\subsection{Breadth First Search}
The \texttt{breadth first} search was one of our cleanest solutions, even though it takes the most memory.  In the Figure ~\ref{fig:bf} the search grows outward from a central location, and then (in the following figure) finds an optimal solution.  Nodes popped: 7,922.  Distance of the path: 1,010 meters.

\begin{figure}\label{fig:bf}
\begin{minipage}[b]{0.3\linewidth} % A minipage that covers half the page
\centering
\includegraphics[width=0.9\linewidth]{bf1.png}
\end{minipage}
\hspace{0.5cm} % To get a little bit of space between the figures
\begin{minipage}[b]{0.3\linewidth}
\centering
\includegraphics[width=0.9\linewidth]{bf2.png}
\end{minipage}
\hspace{0.5cm} % To get a little bit of space between the figures
\begin{minipage}[b]{0.3\linewidth}
\centering
\includegraphics[width=0.9\linewidth]{bf3.png}
\end{minipage}
\caption{Breadth first grows}
\end{figure}

\begin{figure}\label{fig:bfsol}
\begin{center}\centering
\includegraphics[width=0.8\textwidth]{bf4.png}
\caption{Breadth first finds a solution}
\end{center}
\end{figure}

\subsection{Iterative Deepening Search}
The \texttt{iterative deepening} search was the most time-intensive of the group, and in order to complete it we had to significantly reduce the search space.  In addition---unlike other experiments---we moved the flag and the tank close together so that it would only have to search within a radius of 5 squares for the solution.  Our Ruby implementation of \texttt{iterative deepening} would not run past 10 iterations due to time constraints.  Nodes popped: 215.  Distance of the path: 419 meters.

\begin{figure}
\begin{center}\centering
\includegraphics[width=\textwidth]{id1.png}
\caption{Iterative deepening (The start point and the flag have been moved close together in order to find a path in a reasonable amount of time)}
\end{center}
\end{figure}

\subsection{Greedy Best First Search}
The \texttt{greedy best first} algorithm was by far the fastest algorithm that we experimented with.  Its solution, while not optimal, is also one of the best.  See Figure ~\ref{fig:gbf}.  Nodes popped: 114.  Distance of the path: 1,485 meters.

\begin{figure}\label{fig:gbf}
\begin{center}\centering
\includegraphics[width=\textwidth]{gbf.png}
\caption{Greedy best first solution}
\end{center}
\end{figure}


\subsection{$A^*$ Search}
The $A^*$ solution is clean and direct---it seems to follow the most accurate path through the center and then to the destination.  It is also interesting to see how it first considered a route that skirts around the main obstacles by heading `south', but then settles on a path through the middle.  See Figure ~\ref{fig:astar}.  Nodes popped: 190.  Distance of the path: 946 meters.
\par
In addition to a simple $A^*$ search, we also placed a tank in the middle of the game board so as to be able to observe an effect on the search path outcome.  When we placed a tank in the way of the search, the algorithm chose a path that popped 44 nodes and had a distance of 934 meters.

\begin{figure}\label{fig:astar}
\begin{center}\centering
\includegraphics[width=\textwidth]{astar.png}
\caption{$A^*$ solution}
\end{center}
\end{figure}

\section{Adventures in Heuristics}
The obvious heuristic to use on a full-observable map is the straight-line heuristic.  The straight-line is admissible which guarantees us an optimal result.  We didn't bother changing our heuristic (other than to enforce penalties) for the rest of the lab until we started this report.  The consequence of not changing our heuristic until asked to do so is that our alternate heuristics are somewhat contrived, but did provide interesting results.
\par
The first alternate heuristic we tried was a polynomial that took the square of the straight-line distance and added 183 (see Figure ~\ref{fig:poly}).  We hoped that because this heuristic is not admissible that we would get a suboptimal answer---and we did.
\begin{figure}\label{fig:poly}
\begin{center}
\includegraphics[width=\textwidth]{heur1.png}
\caption{Polynomial heuristic $d^2 + 183$}
\end{center}
\end{figure}
The solution looks very much like a \texttt{greedy best first} solution because the heuristic diminishes as a square the closer we get to the goal so that the path-cost is never a competitive factor with the heuristic.  The $A^*$ algorithm ended up choosing the same solution as the \texttt{greedy best first} when using this heuristic.  In the end this mean that we considered 38 nodes, used 37 of them in the final solution and traversed a total of 814.5 meters.
\par
The next alternate heuristic we tried (see Figure ~\ref{fig:circ}) was to find the circular arc distance between our current node and the goal.  This ends up being a multiplicative constants of $\pi$.  I assumed that multiplying everything by $\pi$ would have a null effect since it would be a uniform penalty on all nodes, but in fact it did change the solution quite a bit.
\begin{figure}\label{fig:circ}
\begin{center}
\includegraphics[width=\textwidth]{heur2.png}
\caption{Circular arc distance heuristic}
\end{center}
\end{figure}
This solution was sub-optimal and took the `southern road' below the obstacles rather than passing above them.  This solution evaluated 38 nodes and used 37 of them in the solution which was on-par with the polynomial heuristic.  The total distance traveled was 797.9 meters which is less than the polynomial heuristic above.  The reason that this multiplicative constant ended up affect our final solution is that it affects our heuristic, but not our cost path.  So our total cost for each node was an imbalanced summation of the heuristic and the path cost.
\par
Finally using our straight-line heuristic we came up with an optimal solution that took a lot longer to compute.  See Figure ~\ref{fig:straight}.

\begin{figure}\label{fig:straight}
\begin{center}
\includegraphics[width=\textwidth]{heur3.png}
\caption{Straight-line distance heuristic}
\end{center}
\end{figure}


\section{Application to BZFlag}
In the process of discretizing the map, we discovered that certain sizes are unrealistically large and are therefore unsuitable.  For example, anything over 80 units is almost certain to contain at least one obstacle, and therefore makes the map almost impassable.  On the other hand, choosing a discretization size less than 10 or so produces a map so large ($80 \times 80 = 6400$) that real-time decision making for our Ruby code becomes difficult.
\par
We are probably going to choose the \texttt{greedy best first} search algorithm for future work because of its remarkable speed and (usually) quite satisfactory results.  With regard to measuring edge cost, our weighted penalties of 1.5 times the normal cost of traversing a cell seemed to work well.  For cells that are inside an obstacle, we simply do not treat them as successors of their neighboring cells, and they are therefore ignored.
\par
The straight-line heuristic was the simplest and most effective of the (admittedly arbitrary) alternatives that we chose.  Therefore, it seems a convincing candidate for our future labs.

\section{Tests with Another Group}
\subsection{Our Tests}
All of our tests were run on the default BZFlag map with 24 large obstacles and 28 small.  In most cases, our team was the `green' team on the `east' side of the game area and was attempting to search out the `red' team on the `west' side of the game area.  In the case of \texttt{breadth first} search, the tank started somewhere near the middle of the game area, and targetted a flag on the `east' side.
\par
For each algorithm, the `nodes popped' and path distances were recorded in section ~\ref{sec:search}.  The numbers seem reasonable except in the case of our \texttt{iterative deepening} search, where the nodes popped are significantly lower than expected.  This is in part because our algorithm was so slow that we needed to give it a little boost by putting the tank close to the goal---very close, in fact.  Thus, our measurements in this particular case are not comparable with the other algorithm measurements.

\subsection{Dave Brinton et. al.}
We helped test with Dave Brinton et. al.  Their procedure went smoothly, and in particular, their gnuplot maps were especially good---their maps branched at each cell as it checked successors, showing a line from each cell to its successor cells.  This animation made it very clear what their algorithms were doing.
\par

\begin{itemize}
    \item \texttt{breadth first}	Nodes popped: 1,098, Distance: 1,018 m.
    \item \texttt{depth first}	Nodes popped: 292, Distance: 6,368 m.
    \item \texttt{iterative deepening}	Nodes popped: 3,074,227, Distance: 504 m.
    \item \texttt{greedy best first}	Nodes popped: 54, Distance: 1,049 m.
    \item $A^*$	Nodes popped: 1,028, Distance: 1,047 m.
    \item $A^*$ (with tank) Nodes popped: 1,023, Distance: 1,047 m.
\end{itemize}

For each of the search tests above, we chose the default map, except in the cases of \texttt{depth first} and \texttt{iterative deepening}, where the ``Four Ls'' map was chosen.








%-----------------------------------------------------------
% \addcontentsline{toc}{chapter}{\numberline{}Bibliography}
%\include{biblio}
%-----------------------------------------------------------
\appendix
\chapter{Ruby Code}\label{app:code}
See accompanying archive file (\emph{lab4.tar.gz}) for a complete listing of this lab's Ruby and C code.
%\include{app1}
%\include{app2}
%\include{app5}
%\include{app3}
%-----------------------------------------------------------
\end{document}
