\chapter{Ruby Code}\label{app:code}
Listed below is the partial source of our Ruby search implementation. A complete listing of our project's code was included in the first lab report.  Since much of our code is the same as Lab 1, we have not duplicated it here.
\par

\section{search.rb}
\begin{verbatim}
bzrequire 'lib/agent/basic'
bzrequire 'lib/collection/stack'
bzrequire 'lib/collection/queue'
bzrequire 'lib/collection/priority_queue'
require 'set'

module BraveZealot
  module Agent
    class Search < Basic
      def to_gnuplot
        str = @hq.map.to_gnuplot
        last = nil
        s = @n
        if $options.debug then
          get_log.each do |n|
            if !last.nil?  then
              str += "set arrow from #{last.center.x}, #{last.center.y} " +
                     "to #{n.center.x}, #{n.center.y} nohead lt 5\n"
              str += "plot '-' with lines\n"
              str += " 0 0 0 0\n"
              str += "e\n"
              str += "pause 0.005000\n"
            end
            last = n
          end
        end
        puts "Flag at #{@hq.map.goal.to_coord.inspect} -> Our solution is " +
             "#{s.to_coord.inspect} (#{s.actual_cost} -> " +
             "#{s.predecessors.size + 1}) Nodes popped: #{get_log.size}"
        last = s
        list = s.predecessors
        s.predecessors.each do |n|
          if !last.nil?  then
            str += "set arrow from #{last.center.x}, #{last.center.y} to " +
                   "#{n.center.x}, #{n.center.y} nohead lt 1\n"
            str += "plot '-' with lines\n"
            str += " 0 0 0 0\n"
            str += "e\n"
            str += "pause 0.005000\n"
          end
          last = n
        end
        str
      end

      def save_gnuplot
        f = File.new($options.gnuplot_file,'w')
        f.write(to_gnuplot)
        f.close
        puts "Finished!"
        $stdout.flush
      end

      def log(n)
        @log ||= []
        @log << n
      end

      def get_log
        @log ||= []
      end
    end

    class UninformedSearch < Search

      protected

      def search(init, fringe, limit = 0)
        limited = false
        closed = Set.new
        fringe = fringe.insert(init)
        loop do
          # puts closed.size
          # puts
          if fringe.empty?
            puts "Done search: #{limit}"
            return (limited ? :limited : false)
          end
          node = fringe.remove
          log(node)
          if node.goal?
            puts "Done search: found"
            @n = node
            return node
          end
          if !closed.include?(node)
            closed.add(node)
            size = node.predecessors.size
            if size < limit
              node.succ.each do |n|
                n2 = n.clone
                n2.predecessors = ( node.predecessors.clone ) << node
                fringe.insert(n2)
              end
            else
              limited = true
            end
            # fringe.insert_all(node.succ)
          end
        end
      end
    end

    class DepthFirstSearch < UninformedSearch
      def start
        init = @hq.map.chunk_at_point(@tank.x, @tank.y)
        fringe = Collection::Stack.new
        search(init, fringe)
        save_gnuplot
      end
    end

    class DepthLimitedSearch < UninformedSearch
      def start
        init = @hq.map.chunk_at_point(@tank.x, @tank.y)
        iter_search(init)
        save_gnuplot
      end

      def iter_search(init)
        i = 1
        while (search([init], i)) == :limited
          puts "Iter depth: #{i}"
          i += 1
        end
      end

      def search(list, limit)
        (return :limited) if list.size > limit
        me = list.first
        return me if me.goal?
        puts "I am chunk #{me.x}, #{me.y}"
        log(me)
        result = false
        me.succ.each do |s|
          unless list.include?(s)
            puts ("  " * (list.size - 1)) + "successor #{s.x}, #{s.y}"
            s.predecessors = list.clone
            result = search([s] + list, limit)
            # p result.class
            # If it's a goal node, return right away
            return (@n = result) if result.is_a?(Chunk)
          end
        end
        # The last result will either be 'false' or :limited, and
        # we should return it in either case
        return result
      end

    end

    class BreadthFirstSearch < UninformedSearch
      def start
        init = @hq.map.chunk_at_point(@tank.x, @tank.y)
        fringe = Collection::Queue.new
        search(init, fringe)
        save_gnuplot
      end
    end

    class InformedSearch < Search
      def start
        init = @hq.map.chunk_at_point(@tank.x, @tank.y)
        fringe = Collection::PriorityQueue.new
        @n = search(init, fringe)
        save_gnuplot
      end

      def search(init, fringe)
        closed = Set.new
        fringe.insert(init)
        loop do
          return false if fringe.empty?
          node = fringe.remove
          return node if node.goal?
          if !closed.include?(node)
            log(node)
            closed.add(node)
            # This is where informed searches are different we must evaluate
            # a priority before pushing onto the priority queue
            node.succ.each do |n|
              n2 = n.clone
              n2.g = node.g + (node.center.vector_to(n.center).length)
              n2.predecessors = ( node.predecessors.clone ) << node
              fringe.insert(n2)
            end
          end
        end
      end
    end

    class GreedyInformedSearch < InformedSearch
      def search(init, fringe)
        closed = Set.new
        fringe.insert(init)
        loop do
          return false if fringe.empty?
          node = fringe.remove
          return node if node.goal?
          if !closed.include?(node)
            log(node)
            closed.add(node)
            # This is where informed searches are different we must evaluate
            # a priority before pushing onto the priority queue
            node.succ.each do |n|
              n2 = n.clone
              n2.predecessors = ( node.predecessors.clone ) << node
              fringe.insert(n2)
            end
          end
        end
      end
    end
  end
end
\end{verbatim}


\section{stack.rb}
\begin{verbatim}
bzrequire 'lib/collection/base'

module Collection
  class Stack < Base
    def insert(e)
      @data.push(e)
      self
    end
    def remove
      @data.pop
    end
  end
end
\end{verbatim}

\section{queue.rb}
\begin{verbatim}
bzrequire 'lib/collection/base'

module Collection
  class Queue < Base
    def insert(e)
      @data << e
      self
    end
    def remove
      @data.shift
    end
  end
end
\end{verbatim}

\section{priority\textunderscore queue.rb}
\begin{verbatim}
bzrequire 'lib/collection/containers/priority_queue'

module Collection
  class PriorityQueue < Base
    def initialize
      @pq = Containers::PriorityQueue.new()
    end

    def insert(e)
      # We can just make sure that each Chunk can return a
      # priority for us.  We negate the priority so that we
      # get a priority that returns the 'smallest' priority
      # first instead of the 'biggest' priority first.
      @pq.push(e, -1*e.priority)
    end

    def remove
      @pq.pop
    end

    def size
      @pq.size
    end

    def empty?
      @pq.empty?
    end
  end
end
\end{verbatim}