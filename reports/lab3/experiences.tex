\chapter{Experiences and Time Spent}\label{chap:exp}
\section{Duane's Summary}
Michael took the lead on this project by implementing the two most challenging search algorithms of those assigned---$A^*$ and \texttt{greedy best first}.  My portion was to implement \texttt{depth first}, \texttt{breadth first}, and \texttt{iterative deepening} search.  I also spent some time refactoring from our last week's work since many of the shortcuts we took before the deadline compromised the quality of the code.
\par
The \texttt{depth first} and \texttt{breadth first} algorithms were fairly straight-forward to implement---the most interesting part was the insight (from Russell and Norvig) that these two algorithms can both be implemented with the same code, but different queueing mechanisms.  With that insight, we chose to create a single search algorithm that would take a \texttt{queue} or \texttt{stack} object as a parameter.  \texttt{Breadth first} search corresponded with the \texttt{queue} object and \texttt{depth first} with the \texttt{stack}.
\par
Implementing the \texttt{iterative deepening} search was less straight-forward.  While on the surface it appears that a depth-limited search inside an iteratively deepening outer loop is sufficient to implement \texttt{iterative deepening}, it did not appear to be the case.  Instead, we needed each node to keep track of its predecessors and to consider those predecessors as the ``closed set'' during each inner iteration.  This is not the same as depth-limited search inside an outer loop because a depth-limited search will retain its ``closed set'' when in fact it should be forgetting certain paths that it has already tried (but again, it should not be revisiting nodes in its line of predecessors).
\par
I spent about 16 hours on this project, divided as follows:
\begin{itemize}
    \item 6 hours cleaning up and refactoring code from last week
    \item 3 hours implementing the search code mentioned above
    \item 2 hours debugging and passing off with another team in the CS Sports lab.
    \item 5 hours writing this report and formatting it in \LaTeX
\end{itemize}

\section{Michael's Summary}
This lab was especially fun because once again we got to see our agents take action in the bzflags world.  We now had the ability to see what our agents were 'thinking' by generating PDF maps of how they saw the world and we also had the ability to watch them act by joining the bzflag world as an observer.  This made it much more insteresting to work on because we could see immediate results.
\par
Duane was the first to make serious progress with his $A*$ module written in C.  This gave Jon and I a good place to start implementing out parts.  I focused on implementing the path-following code and using potential fields.  The original implementation would always place an attractive potential field on the next node in the solution path, but that made our agents very slow and tentative because the individual path segments were very small. This gave rise to the idea of placing the attractive field further ahead.  This eventually turned into the algorithm that places the Potential Field as far ahead as possible before encountering a turn.
\par
The night I stayed up late working on $A^*$ I made a costly error of assuming that the problem was in the execution of the algorithm.  This meant that I spent hours of time looking at the algorithm trying to figure out what was wrong before I realized that the problem was my bookkeeping.  The fringe list was stored with references to each chunk on the map so when I re-encountered that same node its path-cost and heuristic were updated and changed which affected when it would be popped off.  Once I found the problem the algorithm magically started to work!
\par
In the end I spent a total of about 20 hours split across the following tasks:
\begin{itemize}
    \item 4 hours of re-working code from last lab to try to make our ``smart'' agents notice when the position of the flag has changed.
    \item 2 hours playing with algorithms to find if a point is inside an obstacle
    \item 4 hours of discretizing the map
    \item 3 hours implementing $A^*$
    \item 1 hour implementing \texttt{greedy best first}
    \item 2 hours finalizing and integrating code with Duane
    \item 4 hours passing off and preparing report material
\end{itemize}

\section{Jon's Summary}